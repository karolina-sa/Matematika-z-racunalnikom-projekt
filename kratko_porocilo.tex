\documentclass[a4paper, 10pt]{article}

\usepackage[slovene]{babel}
\usepackage[utf8]{inputenc}
\usepackage[T1]{fontenc}
\usepackage{lmodern}
\usepackage{amsmath}
\usepackage{amsfonts}
\usepackage{amssymb}
\usepackage{enumitem}
\usepackage{epsdice}
\usepackage{array}
\usepackage[table]{xcolor}
\usepackage{makecell}
\usepackage{hyperref}

\newcolumntype{P}[1]{>{\centering\arraybackslash}p{#1}}

\begin{document}

\title{\textbf{\LARGE{Kratko poročilo}}}
\author{Karolina Šavli}
\date{5.\ 4.\ 2024}

\maketitle

% =======================================================================================================================

\noindent V projektni nalogi bom pod mentorstvom GEN-I analizirala in napovedovala odjem električne energije 
\textbf{gospodinjskih odjemalcev} (v analizo niso vključeni samooskrbni odjemalci, torej tisti, 
ki imajo sončno elektrarno). Gre za obravnavo časovne vrste; odjema električne energije skozi čas. \\

\noindent Glavni cilj projekta je sestaviti metodo, model, ki bo napovedal odjem za celotni naslednji dan (za naslednjih 24 ur), 
kjer bomo upoštevali dejavnike, ki se nam zdijo pomembni za napoved (temperatura, sevanje). \\

\noindent Podjetje GEN-I je pripravilo tabelo podatkov, sestavljeno iz sedem stolpcev:
\begin{itemize}
    \item  \texttt{DateTimeStartUTC}: univerzalni koordinirani čas,
    \item  \texttt{DateTimeStartCET}: srednjeevropski čas,
    \item  \texttt{Odjem ACT}: neto odjem električne energije v kWh,
    \item  \texttt{Temperatura ACT}: dejanska temperatura, 
    \item  \texttt{Temperatura FC}: napovedana temperatura,
    \item  \texttt{Sevanje ACT}: dejansko sevanje in
    \item  \texttt{Sevanje FC}: napovedano sevanje. 
\end{itemize}

\begin{figure}[h!]
    \centering
    \caption{Podatki (vir: GEN-I)}\par\medskip
    \includegraphics[width=\textwidth]{tabela.png}
\end{figure}

\noindent Uporabljala bom vse stolpce, razen stolpca \texttt{DateTimeStartUTC}, saj v 
okviru časa ključen stolpec \texttt{DateTimeStartCET}.  \\

\noindent Odjem je podan za odboje od $1$. novembra $2021$ do $12$. februarja $2024$, 
na vsakih $15$ minut in obsega $80063$ enot podatkov. \\

\noindent Poglejmo si odjem za leti $2022$ in $2023$. Zaradi boljše preglednosti sem podatke povprečila na dnevni ravni.

\begin{figure}[h!]
    \centering
    \caption{Podatki (Odjem električne energije, 2022-2023)}\par\medskip
    \includegraphics[width=0.95\textwidth]{output.png}
\end{figure}

\noindent Z grafa je razvidna sezonskost; očitno večji odjem v zimskih mesecih, kar je pomembno pri 
načrtovanju in upravljanju proizvodnje in distribucije električne energije. 
V zimskih mesecih je odjem višji, zaradi ogrevanja, razsvetljave, saj se število ur 
dnevne svetlobe podaljša in nasploh se poveča uporaba električnih aparatov, kot 
so grelniki, sušilniki in podobno.  \\

\noindent Torej na odjem električne energije ključno vpliva del leta in s tem temperature, 
ki jih imamo podane v tabeli. Imamo pa tudi podatek 
sevanja, ki pričakujem, da ne bo tako pomemben, saj v moji analizi niso 
vključeni odjemalci, ki imajo sončno elektrarno. Mogoče pa bo vseeno 
imel nekaj vpliva, saj močna sončna svetloba povzroči povečano porabo 
električne energije za hlajenje, ker se ljudje zatekajo k napravam za 
hlajenje prostorov. \\

\noindent Analiza in napovedovanje bo izvedeno v programskem jeziku Python, na zgornjih podatki, ki jih bom 
mogoče dopolnila, še s kakšnin novim relevantnim dejavnikom, ki ga bom najverjetneje pridobila
s strani \href{https://ot.borzen.si/Domov/Podatki-trga/Koli%C4%8Dine-in-zneski-izravnave}{Borzen}. \\


\end{document}