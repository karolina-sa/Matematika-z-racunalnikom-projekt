\documentclass[a4paper, 10pt]{article}

\usepackage[slovene]{babel}
\usepackage[utf8]{inputenc}
\usepackage[T1]{fontenc}
\usepackage{lmodern}
\usepackage{amsmath}
\usepackage{amsfonts}
\usepackage{amssymb}
\usepackage{enumitem}
\usepackage{epsdice}
\usepackage{array}
\usepackage[table]{xcolor}
\usepackage{makecell}
\usepackage{hyperref}

\usepackage{geometry}
\geometry{
 a4paper,
 total={170mm,257mm},
 left=25mm,
 top=25mm,
 }

\newcolumntype{P}[1]{>{\centering\arraybackslash}p{#1}}

\begin{document}

\title{\textbf{\LARGE{Model napovedi odjema električne energije}} \\ Matematika z računalnikom 2023/24}
\author{Karolina Šavli}
\date{Maj 2024}

\maketitle

% =======================================================================================================================

\noindent V projektni nalogi bom pod mentorstvom GEN-I analizirala in napovedovala odjem električne energije 
\textbf{gospodinjskih odjemalcev} (v analizo niso vključeni samooskrbni odjemalci, torej tisti, 
ki imajo lastno sončno elektrarno). Obravnavala bom časovno vrsto; odjem električne energije skozi čas. \\


\end{document}